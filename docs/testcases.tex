\documentclass{article}
\usepackage{natbib}
\usepackage{amsmath}
\usepackage{bm}

\newcommand{\Jemmacomment}[1]{\emph{\bfseries Jemma says: #1}}

\title{Advanced Parallel in Time Algorithms for Partial Differential Equations: Testcases}
\author{J. Shipton}

\begin{document}

\maketitle

This document collates and describes the test cases for the Weather
and Climate Prediction and Fusion Modelling use cases for the APinTA
PDEs Excalibur project. We present the information required to
reproduce the tests and model performance metrics. Where possible, we
will reference published material and not repeat the details here, but
will add any further information as required.

\section{Weather and Climate Prediction}

The tests described in the following subsections comprise test suites 1-4.

\subsection{Suite 1: shallow water testcases}
This suite consists of linear and nonlinear shallow water tests on the
plane and sphere. Linear tests are useful for benchmarking the
exponential integrators we are developing since these require
splitting the equations into linear and nonlinear parts.

\subsubsection{1.1: Geostrophic balance, linear equations}
The ability of a model to maintain geostrophic balance is of key
importance for weather and climate prediction. The solid body rotation
test (number 2 in \citet{williamson1992standard}) is initialised in
geostrophic balance and is commonly used to demonstrate this
property. A linearised version is described in
\citet{weller2012computational}, section 4b. Solutions are initialised
in a solid body rotation solution and solved for 5 days, and the $L_2$
norm of the solutions are compared relative to the initial condition.

\subsubsection{1.2: Gaussian initial condition, linear equations}
Initial conditions comprised of Gaussian bumps added to a flow at rest
provide a way to analyse numerical dispersion errors. They have been
used for the linear shallow water equations on the f-plane and
f-sphere \citep{schreiber2018beyond, schreiber2019parallel}.

\subsubsection{1.3: Flow over a mountain, nonlinear equations}
The flow over a mountain test case specified in
\citet{williamson1992standard} has been the subject of much discussion
since the initial conditions are not a solution of the shallow water
equations and the wave becomes unstable at long times. Solutions are
evaluated relative to reference solution after 15 days and inspected
for stability at 50 days.

\subsubsection{1.4: barotropic instability, nonlinear equations}
The unstable barotropic jet described in \citet{galewsky2004initial}
is a commonly used testcase for the nonlinear shallow water
equations. The instability causes the jet to roll up into vortices the
pattern of which has been reproduced by many models, although there is
no analytical solution to compare to. The complex flow presents a
challenge to numerical models and mimics the multiscale, nonlinear
behaviour found in the real atmosphere. The test has been used to
investigate the best way to treat the Coriolis term, with
\citet{schreiber2019exponential} finding that they achieve the best
wallclock-time to error when the incorporate the Coriolis term into
the nonlinear operator in their REXI computation. Solution quality
is inspected compared to reference solutions.

\subsection{Suite 2: vertical slice testcases}
This suite concerns vertical slice testcases for the nonhydrostatic
compressible Euler equations. The testcases are a subset of those
described in \citet{cotter2022compatible}.  They are all for
comparison to published reference solutions, with an eye towards
numerical stability and solution quality, as well as convergence
rate of iterative PinT algorithms.

Briefly, the test problems
are:
\begin{itemize}
\item 2.1: Gravity waves (section 3.1 of the reference),
\item 2.2: Flow over Mount Agnesi (section 3.3 of the reference),
\item 2.3: Flow over Mount Sch\"ar (section 3.4 of the reference).
\end{itemize}

\subsection{Suite 3: 3D testcases}
This suite concerns 3D solutions of the nonhydrostatic (in our case)
compressible Euler equations, using a subset of the test cases proposed
in \citet{ullrich2012dynamical}, namely:
\begin{itemize}
\item 3.1: Mountain wave on a small planet (test case 2-1 of the reference)
\item 3.2: Gravity wave on a small planet (test case 3-1 of the reference)
\item 3.3: Dry baroclinic wave on a reduced planet (test case 4-1-2 of
  the reference)
\end{itemize}
In all cases the solutions are compared to reference solutions with an
eye towards numerical stability and solution quality, as well as
convergence rate of iterative PinT algorithms.

\subsection{Suite 4: Challenging testcases}
This suite comprises testcases selected to understand how PinT
algorithms respond to two specific challenges highlighted in the
ExCALIBUR call, namely 1) models with non-differentiable switches in
(i.e. ``physics''), and 2) models with diffusive components.
\begin{itemize}
\item Challenge 1 is addressed by testcase 4.1, which is the moist rotating
shallow water ``tropical cyclone'' testcase of \citet{ferguson2019assessing}.
\item Challenge 2 is addressed by testcase 4.2, which is the vertical slice
falling bubble testcase in Section 3.2 of \citet{cotter2022compatible}.
\end{itemize}

In all cases the solutions are compared to reference solutions with an
eye towards numerical stability and solution quality, as well as
convergence rate of iterative PinT algorithms.

\section{Fusion Modelling}

This test suite is based on model problems currently being
investigated or related to the UKAEA-led NEPTUNE project
(\emph{NEutrals \& Plasma TUrbulence Numerics for the Exascale}),
which aims to create exascale-level software capable of modelling the
plasma edge of a thermonuclear tokamak plasma. The two test cases
below relate to

\subsection{Anisotropic heat equation}

One of the key challenges in the construction of these models is to accurately
determine the deposition of heat on the interior surface of the tokamak. This is
numerically highly challenging: due to the ionized nature of plasma and the
presence of a strong magnetic field, the dynamics of particles of plasma and the
associated energy dissipation is very non-isotropic. The charged particles,
e.g., electrons and ions, move rapidly in tight spiral orbits, known as
gyro-orbits, along the magnetic field lines, but tremendously slowly along the
normal direction. This means that thermal conductivity along field lines is far
greater than that perpendicular to them: i.e.
$\kappa_{\parallel} \gg \kappa_{\perp}$, which can typically be 10 orders of
magnitude in difference. In order to identify suitable numerical methods for
modelling these problems, and to examine the numerical diffusion and dispersion
of the solution field, this test cases focuses on a simplified heat transport
model of the form
%
\[
  \frac{3}{2} n \frac{\partial T}{\partial t} = \nabla \cdot \big(\bm{\kappa}_c
  \cdot \nabla T \big) + \nabla \cdot [\kappa_{\wedge}\bm{b}\times\nabla T]
\]
%
where $n$ is the number density of electrons (approximately the same to the
number density of ions), $T$ is the temperature, $\bm{b}$ is the magnetic field,
and $\bm{\kappa}_c$ is the thermal conductivity tensor
\[
  \bm{\kappa}_c = (\kappa_{\parallel}-\kappa_{\perp}) (\bm{b}\otimes \bm{b}) +
  \kappa_{\perp}\bm{I}.
\]
This problem can be considered either in two dimensions (in which case
$\kappa_\wedge = 0$) or more complex three-dimensional arrangements.

\subsection{Plasma model}

Another key challenge in delivering a successful magnetically-confined tokamak
fusion reactor is to understand and control the turbulence that arises in the
edge region where hot plasma encounters cold neutral gases and the wall of the
tokamak. A dominant mechanism in this area is drift-wave turbulence.  To
investigate an inherently turbulent model, we proposed to use a common test case
for this problem, based on the Hasegawa-Wakatani
equations~\citep{wakatani1984collisional}. This takes the form
%
\begin{align*}
  \frac{\partial\zeta}{\partial t} + [\phi, \zeta] &= \alpha (\phi - n) \\
  \frac{\partial n}{\partial t} + [\phi, n] &= \alpha (\phi - n) - \kappa \frac{\partial\phi}{\partial y}
\end{align*}
%
where $\zeta$ is the vorticity, $n$ is the perturbed number density, $\phi$ is
the electrostatic potential, and
%
\[
  [a,b] = \frac{\partial a}{\partial x} \frac{\partial b}{\partial y} -
  \frac{\partial a}{\partial y} \frac{\partial b}{\partial x}
\]
%
is the canonical Poisson bracket operator. The vorticity and electrostatic
potential are related through the Poisson equation $\nabla^2\phi =
\zeta$. $\alpha$ is the adiabiacity operator (taken to be constant in this
solver), and $\kappa$ is the background density gradient scale length.  The
transitional behaviour of this test cases poses a significant modelling
challenge.

\bibliographystyle{plainnat}
\bibliography{references}

\end{document}
